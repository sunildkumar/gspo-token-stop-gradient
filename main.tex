\documentclass{article}
\usepackage{graphicx}
\usepackage[margin=1in]{geometry}
\title{gspo-token}
\author{sdkumar }
\date{August 2025}

\begin{document}
Goal: Understand the need of stop gradient for GSPO-token. \\

\paragraph{Observation 1.}
Let $f$ be a differentiable function, $x$ a scalar or vector input, and $\mathrm{sg}[\cdot]$ the stop-gradient operator. 
The stop-gradient operator returns the same numerical value as its argument but is treated as a constant during backpropagation. 
Formally,
\[
\frac{d}{dx} \,\mathrm{sg}\!\big(f(x)\big) = 0.
\] \\



\paragraph{Observation 2.}
Assume we had no stop gradient. Then we have
\[
s_{i,t}(\theta) 
= s_{i}(\theta) \,
\frac{\pi_{\theta}(y_{i,t} \mid x, y_{i,<t})}
     {\pi_{\theta}(y_{i,t} \mid x, y_{i,<t})}.
\]

Taking the derivative with respect to the parameters:
\[
\frac{d}{d\theta} s_{i,t}(\theta) 
= \frac{d}{d\theta} s_{i}(\theta) \,
\frac{\pi_{\theta}(y_{i,t} \mid x, y_{i,<t})}
     {\pi_{\theta}(y_{i,t} \mid x, y_{i,<t})}.
\]

Note that the fraction on the right is constant (equal to $1$), so its derivative is $0$. 
This results in us pushing gradient through the importance weight $s_i(\theta)$, which is exactly what we want to prevent:
\[
\frac{d}{d\theta} s_{i,t}(\theta) 
= \frac{d}{d\theta} s_{i}(\theta).
\]

In GSPO-token, $s_i(\theta)$ is meant to be a stable
sequence-level importance ratio that corrects for off-policy sampling. If gradients are allowed
to update $s_i(\theta)$ here, we are no longer treating it as a fixed correction factor.  Instead,
we are optimizing it directly. I believe the authors claim this can destabilize training and undermine
the intended off-policy correction.

\paragraph{Observation 3:}
Consider the version with stop gradient.
In GSPO-token, they define
\[
s_{i,t}(\theta) =
\mathrm{sg}\!\left[ s_i(\theta) \right] \cdot
\frac{\pi_\theta(y_{i,t} \mid x, y_{i,<t})}
{\mathrm{sg}\!\left[ \pi_\theta(y_{i,t} \mid x, y_{i,<t}) \right]}.
\]
Applying Observation~1, both $\mathrm{sg}[s_i(\theta)]$ and $\mathrm{sg}[\pi_\theta(\cdot)]$
are constants with respect to $\theta$ during backpropagation. Differentiating gives
\[
\frac{d}{d\theta} s_{i,t}(\theta)
= \frac{\mathrm{sg}[s_i(\theta)]}
       {\mathrm{sg}[\pi_\theta(y_{i,t} \mid x, y_{i,<t})]}
  \cdot \frac{d}{d\theta} \pi_\theta(y_{i,t} \mid x, y_{i,<t}).
\]
Recognizing that $\frac{1}{\mathrm{sg}[\pi_\theta]}\,\frac{d}{d\theta} \pi_\theta
= \frac{d}{d\theta} \log \pi_\theta$, we can write
\[
\frac{d}{d\theta} s_{i,t}(\theta)
= \mathrm{sg}[s_i(\theta)] \cdot
  \frac{d}{d\theta} \log \pi_\theta(y_{i,t} \mid x, y_{i,<t}).
\]
This shows that the sequence-level importance ratio $\mathrm{sg}[s_i(\theta)]$
acts purely as a fixed multiplier on the token-level log-probability gradient.




\end{document}
